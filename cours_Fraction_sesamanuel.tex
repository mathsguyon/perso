%\listfiles
\PassOptionsToPackage{table}{xcolor}
\PassOptionsToPackage{svgnames}{xcolor}
\documentclass[nocrop]{sesamanuel_guyon}
\renewcommand\PrefixeCorrection{Corrections/}
\usepackage{etex}
\usepackage{ProfCollege}
%\usepackage{pstcol,pst-grad}%

\NewThema{N}{n}{nombres\\\& calculs}{Nombres\\\& calculs}{LES BASES}{red}{red!50}

%\NewThema{E}{e}{espace \&\\géométrie}{Espace \&\\géométrie}{ESPACE \&\\ GEOMETRIE}{DodgerBlue}{DodgerBlue!50}
%
%\NewThema{D}{d}{organisation et \\et gestion de données}{Organisation et \\et gestion de données}{ORGANISATION ET\\GESTION DE DONNÉES}{violet}{violet!50}
%
%\NewThema{M}{m}{grandeurs\\et mesures}{Grandeurs\\et mesures}{GRANDEURS\\ET MESURES}{Green}{Green!50}
%
%\NewThema{A}{a}{algorithmique \&\\programmation}{Algorithmique \&\\ programmation}{ALGORITHMIQUE \&\\PROGRAMMATION}{orange}{orange!50}

\renewcommand\ListeMethodesThemes{{n}{N},{d}{D},{g}{G},{m}{M},{a}{A}}


\newcommand{\R}{\mathbb {R}}
%\newcommand{\C}{\mathds {C}}
\newcommand{\N}{\mathbb {N}}
\newcommand{\Z}{\mathbb {Z}}
\newcommand{\Q}{\mathbb {Q}}
\newcommand{\e}{\mathrm {e}}
\newcommand{\D}{\mathbb {D}}
\begin{document}
	
	\frontmatter
	\pagenumbering{gobble}
	%\include{1ere_de_couverture_5e_22-23}
	
	\mainmatter
	
	\themaN
	
	\chapter{Calculer\\ avec les fractions}
	\label{S01}
	%%%%%%%%%%%%%%%%%%%%%%%%%%%%%%%%%%%%%%
	\begin{autoeval}
	\begin{itemize}
		\begin{multicols}{2}
		\item Simplifier $\dfrac{-18}{-64}=$ \BoiteRapido{}
		\item	$2-\dfrac{5}{12}=$	\BoiteRapido{}
		\item	$\dfrac{3}{4}+\dfrac{7}{10}=$ \BoiteRapido{}
		\item	$\dfrac{3}{4}\times \dfrac{7}{10}=$ \BoiteRapido{}
		\item	$\dfrac{12}{-25}\times \dfrac{-45}{18}=$ \BoiteRapido{}
		\item	$\dfrac{3}{4}\div \dfrac{7}{10}=$ \BoiteRapido{}
		\item	$\dfrac{2}{3}\div 3=$ \BoiteRapido{}
		\item	$\dfrac{3-\dfrac{2}{3}}{3+\dfrac{2}{3}}=$ \BoiteRapido{}
		\item	$\dfrac{3}{4}-\dfrac{1}{4}\times \dfrac{2}{3}=$ \BoiteRapido{}
		\item	Raoul mange les $\dfrac{3}{7}$ d'une pizza.\\
		Simone dévore le tiers du reste.\\
		Quelle part reste-t-il à Hector ?\BoiteRapido{}
		\end{multicols}
	\end{itemize}
	\end{autoeval}
	
	
	\begin{prerequis}
		\small
			\begin{itemize}
		\item	Connaître la définition et propriétés élémentaires d'une fraction.\par
\item	Savoir ajouter deux fractions.\par
	\item Savoir multiplier deux fractions.\par
	\item Savoir diviser deux fractions. \par
	\item Savoir mener un calcul avec des fractions. \par
	\item  Savoir calculer la fraction ou le pourcentage d'un nombre. \par
	 \item Savoir résoudre un problème utilisant les fractions.\par	
		\end{itemize}
	\end{prerequis}
	
\bigskip


	\begin{debat}[Débat : Pourquoi étudier les opérations avec les fractions ?]
\subsection*{Les entiers naturels :}		En primaire, vous avez travaillé avec des nombres entiers. Les premières opérations étudiées (addition et multiplication) sont dîtes "stables" avec les entiers.
		Si on ajoute ou si on multiplie deux entiers naturels, on obtient un entier naturel.\\
		Cet ensemble des entiers naturels, on peut l'écrire : $\big\{0;1;2;3;\ldots \big\}$ comme au collège, mais à partir du lycée, on lui donnera un nom : $\mathbb {N}$. Le chapitre 3 est consacré à ces nouvelles définitions.\\
\subsection*{Les entiers relatifs :}	
Avec la soustraction, 	on se rend compte que les entiers naturels ne suffisent plus pour toutes les opérations.\\
Avec $3-2=1$ , la différence de deux entiers naturels reste un entier naturel
mais la soustraction $2-3$ ne donne pas un entier naturel.
La soustraction n'est donc pas stable avec $\mathbb {N}$.\\
Pour s'en sortir, il faut définir un nouvel ensemble, les entiers relatifs, qu'on va appeler : $\mathbb {Z}$.\\
Il contient tous les entiers naturels, et on rajoute tous leurs opposés : $\mathbb {Z}=\{\ldots ; -3;-2;-1;0;1;2;3; \ldots\}$.\\
Vous avez appris à ajouter, soustraire et multiplier les entiers relatifs au collège.\\
\subsection*{Les décimaux :}		
La difficulté suivante vient avec la division. Avec $2 \div 1=2$ , le quotient de deux entiers donne un entier.\\
Mais $1 \div 2$ ne donne pas un nombre entier.\\
Il a donc fallu créer une nouvelle famille de nombre, avec la numération décimale. On écrit : $1 \div 2 = 0,5$.\\
Ce nouvel ensemble des décimaux  s'appelle $\mathbb{D}$.\\
Vous avez appris en primaire à gérer les quatre opérations avec les décimaux.\\
\subsection*{Les rationnels :}		
Mais les problèmes ne s'arrêtent pas là avec la division.\\
Car le quotient $1 \div 3$ ne donne pas un nombre décimal. La démonstration sera faîte dans le chapitre 3.\\
$1 \div3\approx 0,3333\ldots$ mais il est impossible de trouver un décimal exactement égal à $1\div 3$.\\
Ce nombre existe mais n'a pas d'écriture décimale. Comment s'en sortir ?\\
Inventer une nouvelle notation : $\dfrac{1}{3}$ et un nouvel ensemble qui contienne ces nombres. On appelle cet ensemble celui des rationnels, on le nomme : $\mathbb{Q}$.\\
L'objectif de ce chapitre est donc de vérifier que vous savez opérer avec des éléments de cet ensemble, c'est à dire vérifier que vous savez manipuler les fractions dans des calculs.\\
\subsection*{Illustration}
\begin{center}
	
	\begin{pspicture}(0,-.5)(7,6.5)
	\newrgbcolor{bleu}{0.1 0.05 .5}
	\newrgbcolor{prune}{.6 0 .48}
	\psset{linewidth=1pt,framearc=0.3, linecolor=prune}
	\psframe(1.2,1.2)(4,2.5)
	\uput*[r](2,2.5){\LARGE\prune {$\N$}}
	\uput[u](2,1.5){\bleu {$0$}}
	\uput[d](2.9,1.9){\bleu {$2^3$}}
	\uput[l](3.8,2){\bleu {$15$}}
	\psframe(.9,.9)(5.5,3.5)
	\uput*[r](2,3.5){\LARGE\prune {$\Z$}}
	\uput[r](2.3,2.9){\bleu {$-1$}}
	\uput[d](4.5,3.2){\bleu {$-12$}}
	\uput[d](4.6,2){\bleu {$-3^4$}}
	\uput[r](2,4){\bleu {$-0,475$}}
	\uput[d](3.4,5.5){\bleu {$-\dfrac{4}{3}$}}
	\psframe[linewidth=1.25pt,framearc=0.3, linecolor=bleu](0,0)(8,6.5)
	\psframe(.6,.6)(6,4.5)
	\uput*[r](2,4.5){\LARGE\prune {$\D$}}
	\psframe(.4,.4)(6.8,5.5)
	\uput*[r](2,5.5){\LARGE\prune {$\Q$}}
	\end{pspicture}
\end{center}	

Le chapitre 3 est entièrement consacré à ces ensembles. Il poursuit même cette histoire avec les nombres qui ne s'écrivent pas sous forme de fraction.\\
mais pas question de spolier ici ce chapitre.\\
On reste focalisé dans ce chapitre sur les fractions.\end{debat}
	
	
	%%%%%%%%%%%%%%%%%%%%%%%%%%%%%%
	%%%%%%%%%%%%%%%%%%%%%%%%%%%%%%
		
	
	%%%%%%%%%%%%%%%%%%%%%%%%%%%%%%%%%%%%
	%%%%%%%%%%%%%%%%%%%%%%%%%%%%%%%%%%%%
	\cours 
	
	
	%%%%%%%%%%%%%%%%%
	\section{Généralités avec les fractions :}
	\subsection{Fractions égales}
	
	
	\begin{propriete}
			On ne change pas la valeur d'une fraction en multipliant son numérateur et son dénominateur par même nombre non-nul.
	\end{propriete}
	
	\begin{exemple*1}
		\begin{align*}
				A&=\dfrac{2}{7}&B&=\dfrac{25}{40}\\
				&=\dfrac{2\times 3}{7\times 3}&&=\dfrac{5\times5}{5\times 8}\\
				&=-\dfrac{6}{21}&&=\dfrac{5}{8}\\
		\end{align*}
	\end{exemple*1}
	
\subsection{Fractions irréductibles :}	
	\begin{definition}
		On appelle fraction irréductible, une fraction dont le dénominateur et le numérateur n'ont pas d'autres diviseurs communs que 1.
	\end{definition}	
\begin{exemple*1}	
		\begin{align*}
				&\dfrac{2}{7}  \text{est irréductible}&\dfrac{21}{14} \text{ n'est pas irréductible}\\
		\end{align*}
		\end{exemple*1}
	
	\begin{remarque}
		Quand deux entiers n'admettent que $1$ comme diviseur commun, on dit qu'ils sont alors \textbf{premiers entre-eux}.\\
	\end{remarque}
\begin{approfondissement}
	
On dira que deux entiers sont premiers entre eux, si leur Plus Grand Diviseur Commun (PGCD) est égal à 1.\\
On écrira par exemple : $PGCD(5;8)=1$ donc $5$ et $8$ sont premiers entre-eux, donc $\dfrac{5}{8}$ est irréductible.\\
Mais $PGCD(4;8)=4$ donc $4$ et $8$ ne sont pas premiers entre-eux, donc $\dfrac{4}{8}$ n'est pas irréductible.
	\end{approfondissement}
	%%%%%%%%%%%%%%%%%%%%%%%%%%%%%%%%%%%%
	%%%%%%%%%%%%%%%%%%%%%%%%%%%%%%%%%%%%
	\exercicesbase
	
	\begin{colonne*exercice}
		
		\begin{exercice} %1
			Exercice 1
		\end{exercice}
		
	%	\begin{corrige}
	%		Corrigé de l'exo 1
	%	\end{corrige}
		
		\bigskip
		
		
		\begin{exercice}[éventuel nom] %2
			Exercice 2
		\end{exercice}
		
	%	\begin{corrige}
	%		Corrigé de l'exo 2
		%\end{corrige}
		
		
	\end{colonne*exercice}
	
	
	%%%%%%%%%%%%%%%%%%%%%%%%%%%%%%%%%%
	%%%%%%%%%%%%%%%%%%%%%%%%%%%%%%%%%%
	\Recreation
	
	\begin{enigme}[Nom de l'énigme]
		
		\partie[nom de la partie]
		
	\end{enigme}
	
	
%	\begin{corrige}
%		Corrigé de l'énigme
%	\end{corrige}  
	
	
	%\include{Plans_de_travail_22-23}
	
	\AfficheCorriges[2]
	
	\backmatter
	%\fancyfoot[R]{}
	%\include{4e_de_couverture_22-23}
	
\end{document}