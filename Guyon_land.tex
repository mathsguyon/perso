\documentclass[a4paper,oneside, landscape,12pt]{article}
\usepackage[utf8]{inputenc}
\usepackage[T1]{fontenc}
\usepackage[french]{babel}
%\usepackage{ProfCollege}
\usepackage{amsmath,amsfonts,amsthm,amssymb,amsbsy,amscd,mathrsfs}
\usepackage{lmodern}
\usepackage{fourier}
\usepackage[scaled=0.875]{helvet} % ss
\usepackage{textcomp}
\usepackage{dsfont} % pour les symboles des ensembles
\usepackage[left=1.5cm,right=1.5cm,top=1.5cm,bottom=1.5cm]{geometry}
\usepackage{fancyhdr} 
\fancyhf{}
\setlength{\headheight}{26pt}
\usepackage{tabularx}
\usepackage{qrcode}
\setlength{\footskip}{18pt}
\renewcommand {\headrulewidth}{0pt}
\renewcommand {\footrulewidth}{0pt}
%\usepackage{lastpage}
%\usepackage{makeidx}
%\usepackage{graphicx} % inclure le fichier graphique de pondichery 2010
\usepackage{hyperxmp}
\usepackage[colorlinks=true,pdfstartview=FitV,linkcolor=blue,citecolor=blue,urlcolor=blue]{hyperref}
\pagestyle{fancy}

\usepackage{array}
\usepackage{tabularx}
\usepackage{multirow}
\usepackage{hhline}
\usepackage{mathtools} %pour l'alignement des termes d'une matrice
\usepackage{fltpoint}
\usepackage{xcolor,colortbl}
\usepackage{pstricks,pst-all,pst-plot,pstricks-add,pst-tree,pst-3dplot,pst-func}
\usepackage{ifthen}
\usepackage{calc}
\usepackage{esvect}
\usepackage{pifont}
\setlength{\parindent}{0mm}
\usepackage [alwaysadjust]{paralist}
\usepackage{pgf,tikz}
\usepackage{tkz-tab}
\usepackage{tkz-fct}
\usepackage{tkz-euclide}
\usepackage{xkeyval}
%\usepackage{sesamanuel}

%\usepackage{bclogo}
\usepackage[tikz]{bclogo}
%\setdefaultenum {1.}{a)}{}{}
\hyphenpenalty=10000 
\tolerance=2000
\emergencystretch=1em
\widowpenalty=10000
\clubpenalty=10000
\raggedbottom
\renewcommand*{\tabularxcolumn}[1]{m{#1}} %centrage vertical des cellules d'un tableau tabularx
\newcommand{\Oij}{$\left( {{\mathrm{O}};\vec i,\vec j} \right)$}
\newcommand{\Oijk}{$\left( {{\mathrm{O}};\vec i,\vec j ,\vec k} \right)$}
\newcommand{\euro}{\texteuro{}}
\newcommand{\R}{\mathds {R}}
%\newcommand{\C}{\mathds {C}}
\newcommand{\N}{\mathds {N}}
\newcommand{\Z}{\mathds {Z}}
\newcommand{\Q}{\mathds {Q}}
\newcommand{\e}{\mathrm {e}}
\newcommand{\dd}{\,\mathrm{d}}
\DecimalMathComma
% Mise en forme des algorithmes
\usepackage[french,boxed,titlenumbered,lined,longend]{algorithm2e}
\SetKwIF {Si}{SinonSi}{Sinon}{si}{alors}{sinon\_si}{alors}{fin~si}
\SetKwFor{Tq}{tant\_que~}{~faire~}{fin~tant\_que}
\SetKwFor{PourCh}{pour\_chaque }{ faire }{fin pour\_chaque}
\SetKwInput{Sortie}{Sortie}
\newcommand{\Algocmd}[1]{\textsf{\textsc{\textbf{#1}}}}\SetKwSty{Algocmd}
\newcommand{\AlgCommentaire}[1]{\textsl{\small  #1}}  
%

\newcommand{\fvalid}{\raisebox{1ex}{\rotatebox{180}{$\Rsh$}}} % flèche entrée pour prg casio


%\newcommand{\QCM}[3]{ $\bullet$ \quad #1 \hfill 
%	$\bullet$ \quad #2 \hfill 
%	$\bullet$ \quad #3 }

\newcommand{\QCMc}[4]{   #1 \\ 
	\makebox[2cm][l]{\psframe(.25,.25) \qquad #2 }\hfill 
	\makebox[2cm][l]{\psframe(.25,.25) \qquad #3 }\hfill 
	\makebox[2cm][l]{\psframe(.25,.25) \qquad #4 }} % QCM avec cases à cocher%

\setcounter{secnumdepth}{3}
\makeatletter
\renewcommand\section{\@startsection {section}{1}{\z@}%
	{-3.5ex \@plus -1ex \@minus -.2ex}%
	{2.3ex \@plus.2ex}%
	{\reset@font\bfseries\red\sffamily\scshape}} 

\renewcommand\subsection{\@startsection {subsection}{2}{\z@}%
	{-3.5ex \@plus -1ex \@minus -.2ex}%
	{2.3ex \@plus.2ex}%
	{\reset@font\bfseries\sffamily\scshape\blue}}                                   
\renewcommand\subsubsection{\@startsection {subsubsection}{3}{\z@}%
	{-3.5ex \@plus -1ex \@minus -.2ex}%
	{2.3ex \@plus .2ex}%
	{\reset@font\bfseries\sffamily\scshape\small}}
\makeatother
\setcounter{secnumdepth}{2}

\renewcommand\thesection{\Roman {section}}
\renewcommand\thesubsection{\arabic {subsection}}
%%%% debut macro %%%%
\makeatletter
\renewcommand\theequation{\thesection.\arabic{equation}}
\@addtoreset{equation}{section}
\makeatother
%%%% fin macro %%%%

\newcounter{nexo}           % déclaration du numéro d'exo
\setcounter{nexo}{0}        % initialisation du numero

\newcommand{\exo}{
	\refstepcounter{nexo} 
	\par{\textsf{\textbf{\textcolor{blue}{{\small{Exercice \; \arabic{nexo}}}}}}}}
\newcounter{nquestion}           % déclaration du numéro d'exo
\setcounter{nquestion}{0} 
\newcommand{\question}{
	\refstepcounter{nquestion} 
	\par{\textsf{\textbf{Question \; \arabic{nquestion}}}}}

\newcounter{nactivite}           % déclaration du numéro d'exo
\setcounter{nactivite}{0}        % initialisation du numero

\newcommand{\act}{
	\stepcounter{nactivite} 
	\par{\textsf{\textbf{\textcolor{blue}{{\small{\textsc{activité} \; \arabic{nactivite}}}}}}}}

%%%% debut macro %%%%
\makeatletter
\@addtoreset{nactivite}{section}
\makeatother
%%%% fin macro %%%%

\newrgbcolor{Rfond}{.995 0.945 .985}  
\newrgbcolor{Rougef}{.8 0.1 .2}
\newrgbcolor{Rbord}{.6 0.1 .4}

\newcommand{\ds}{\displaystyle}	%displaystyle
\newcommand{\cg}{\texttt{]}}	%crochet gauche
\newcommand{\cd}{\texttt{[}}	%crochet droit
\newcommand{\pg}{\geqslant}		%plus grand ou égal
\newcommand{\pp}{\leqslant}		%plus petit ou égal

\newcommand{\cadreR}[1]{%
	\psframebox[framesep=.8em,linecolor=black,linewidth=0.25pt,framearc=0.2,fillstyle=solid,fillcolor=Rfond]{\begin{minipage}{\linewidth-1.6em}%
			#1\end{minipage}}}
\newcommand{\cadreB}[1]{%
	\psframebox[framesep=.8em,linecolor=black,linewidth=0.25pt,framearc=0.2]{\begin{minipage}{\linewidth-1.6em}%
			#1\end{minipage}}}

\newcommand{\dem}{
	\medskip
	\par{\textsf{\textsc{{\small{\ding{106}\: démonstration}}}}}%
	\medskip}


\newcommand{\preuve}{
	\medskip
	\par{\textsf{\textsc{{\small{\ding{106}\: preuve}}}}}%
	\medskip} 
\usepackage{pgfplots}
\pgfplotsset{compat=1.15}
\usepackage{mathrsfs}
\usetikzlibrary{arrows}
\usepackage{tabto}
\newcommand{\encadrer}[1]{
	{\setlength{\fboxsep}{2mm}
		\fbox{#1}}}
\usepackage{multicol}
%%%%%%%%%%%%%%%%%%%%%%%%%%%%%%%%%%%%%%%%%%%%%%%%%%%%%%%%%%%%%%%%%%%
\newcommand{\trou}[1]{%
	\iftrou%
	\settowidth{\trouwidth}{#1}%
	\hbox to \trouwidth{\dotfill}%
	\else%
	#1%
	\fi
}
\newcommand{\image}[2]{\includegraphics[width=#1]{#2}}

\tikzstyle{general}         =[line width=0.3mm, >=stealth, x=1cm, y=1cm,line cap=round, line join=round]
\tikzstyle{quadrillage}     =[line width=0.3mm, color=CyanTikz40]
\tikzstyle{quadrillageNIV2} =[line width=0.3mm, color=CyanTikz20]
\tikzstyle{quadrillage55}   =[line width=0.3mm, color=CyanTikz40, xstep=0.5, ystep=0.5]
\tikzstyle{quadrillage33}   =[line width=0.3mm, color=CyanTikz40, xstep=0.33, ystep=0.33]
\tikzstyle{cote}            =[line width=0.3mm, <->]
\tikzstyle{epais}           =[line width=0.5mm, line cap=butt]
\tikzstyle{tres epais}      =[line width=0.8mm, line cap=butt]
\tikzstyle{axe}             =[line width=0.3mm, ->, color=Noir, line cap=rect]

\definecolor{fond1}{rgb}{1,1,.8}%jaune clair
\definecolor{fond2}{rgb}{.8,.9,1}%bleu clair
\definecolor{fond6}{rgb}{1,0.8,0}%orange
\definecolor{LightGray}{gray}{0.9}
\definecolor{LightGreen}{rgb}{0.5,1,0.5}
\definecolor{LightRed}{rgb}{1,0.5,0.5}
\definecolor{A1}                {cmyk}{1.00, 0.00, 0.00, 0.50}%bleu marine
\definecolor{A2}                {cmyk}{0.60, 0.00, 0.00, 0.10}%bleu ciel
\definecolor{A3}                {cmyk}{0.30, 0.00, 0.00, 0.05}
\definecolor{A4}                {cmyk}{0.10, 0.00, 0.00, 0.00}
\definecolor{B1}                {cmyk}{0.00, 1.00, 0.60, 0.40}
\definecolor{B2}                {cmyk}{0.00, 0.85, 0.60, 0.15}
\definecolor{B3}                {cmyk}{0.00, 0.20, 0.15, 0.05}
\definecolor{B4}                {cmyk}{0.00, 0.05, 0.05, 0.00}
\definecolor{C1}                {cmyk}{0.00, 1.00, 0.00, 0.50}
\definecolor{C2}                {cmyk}{0.00, 0.60, 0.00, 0.20}
\definecolor{C3}                {cmyk}{0.00, 0.30, 0.00, 0.05}
\definecolor{C4}                {cmyk}{0.00, 0.10, 0.00, 0.05}
\definecolor{D1}                {cmyk}{0.00, 0.00, 1.00, 0.50}
\definecolor{D2}                {cmyk}{0.20, 0.20, 0.80, 0.00}
\definecolor{D3}                {cmyk}{0.00, 0.00, 0.20, 0.10}
\definecolor{D4}                {cmyk}{0.00, 0.00, 0.20, 0.05}
\definecolor{F1}                {cmyk}{0.00, 0.80, 0.50, 0.00}
\definecolor{F2}                {cmyk}{0.00, 0.40, 0.30, 0.00}
\definecolor{F3}                {cmyk}{0.00, 0.15, 0.10, 0.00}
\definecolor{F4}                {cmyk}{0.00, 0.07, 0.05, 0.00}
\definecolor{G1}                {cmyk}{1.00, 0.00, 0.50, 0.00}
\definecolor{G2}                {cmyk}{0.50, 0.00, 0.20, 0.00}
\definecolor{G3}                {cmyk}{0.20, 0.00, 0.10, 0.00}
\definecolor{G4}                {cmyk}{0.10, 0.00, 0.05, 0.00}
\definecolor{H1}                {cmyk}{0.40, 0.00, 1.00, 0.10}
\definecolor{H2}                {cmyk}{0.20, 0.00, 0.50, 0.05}
\definecolor{H3}                {cmyk}{0.10, 0.00, 0.20, 0.00}
\definecolor{H4}                {cmyk}{0.07, 0.00, 0.15, 0.00}
\definecolor{J1}                {cmyk}{0.00, 0.50, 1.00, 0.00}
\definecolor{J2}                {cmyk}{0.00, 0.20, 0.50, 0.00}
\definecolor{J3}                {cmyk}{0.00, 0.10, 0.20, 0.00}
\definecolor{J4}                {cmyk}{0.00, 0.07, 0.15, 0.00}
\definecolor{FondOuv}           {cmyk}{0.00, 0.05, 0.10, 0.00}
\definecolor{FondAutoEvaluation}{cmyk}{0.00, 0.03, 0.15, 0.00}
\definecolor{FondTableaux}      {cmyk}{0.00, 0.00, 0.20, 0.00}
%\definecolor{FondAlgo}          {cmyk}{0.07, 0.00, 0}
\tikzstyle{mybox} = [draw=black, fill=A4, very thick,
rectangle, rounded corners, inner sep=5pt, inner ysep=10pt]
\tikzstyle{fancytitle} =[fill=red, text=white,rounded corners]
\newcommand{\postit}[2]{%
	\begin{tikzpicture}
		\node [mybox] (box){%
			\begin{minipage}{\linewidth}
				%\centering{#1}
				{#2}
			\end{minipage}
		};
		\node[fancytitle, right=10pt] at (box.north west) {{\bf #1}};
		%\node[fancytitle, rounded corners] at (box.east) {$\clubsuit$};
	\end{tikzpicture}%
	%
}

\newcommand{\postitpp}[3]{%
	\begin{tikzpicture}
		\node [mybox] (box){%
			\begin{minipage}{#1}
				%\centering{#1}
				{#3}
			\end{minipage}
		};
		\node[fancytitle] at (box.north) {{\bf #2}};
		%\node[fancytitle, rounded corners] at (box.east) {$\clubsuit$};
	\end{tikzpicture}%
	%
}